%%%%%%%%%%%%%%%%%%%%%%%%%%%%%%%%%%%%%%%%%%%%%%%%%%%%%%%%%%%%%%%%%%%%%%%%%%%
%%
%% Filename: 	spec.tex
%%
%% Project:	Wishbone scope
%%
%% Purpose:	This LaTeX file contains all of the documentation/description
%%		currently provided with this Wishbone scope core.  It's not
%%		nearly as interesting as the PDF file it creates, so I'd
%%		recommend reading that before diving into this file.  You
%%		should be able to find the PDF file in the SVN distribution
%%		together with this PDF file and a copy of the GPL-3.0 license
%%		this file is distributed under.  If not, just type 'make'
%%		in the doc directory and it (should) build without a problem.
%%
%%
%% Creator:	Dan Gisselquist
%%		Gisselquist Technology, LLC
%%
%%%%%%%%%%%%%%%%%%%%%%%%%%%%%%%%%%%%%%%%%%%%%%%%%%%%%%%%%%%%%%%%%%%%%%%%%%%
%%
%% Copyright (C) 2015, Gisselquist Technology, LLC
%%
%% This program is free software (firmware): you can redistribute it and/or
%% modify it under the terms of  the GNU General Public License as published
%% by the Free Software Foundation, either version 3 of the License, or (at
%% your option) any later version.
%%
%% This program is distributed in the hope that it will be useful, but WITHOUT
%% ANY WARRANTY; without even the implied warranty of MERCHANTIBILITY or
%% FITNESS FOR A PARTICULAR PURPOSE.  See the GNU General Public License
%% for more details.
%%
%% You should have received a copy of the GNU General Public License along
%% with this program.  (It's in the $(ROOT)/doc directory, run make with no
%% target there if the PDF file isn't present.)  If not, see
%% <http://www.gnu.org/licenses/> for a copy.
%%
%% License:	GPL, v3, as defined and found on www.gnu.org,
%%		http://www.gnu.org/licenses/gpl.html
%%
%%
%%%%%%%%%%%%%%%%%%%%%%%%%%%%%%%%%%%%%%%%%%%%%%%%%%%%%%%%%%%%%%%%%%%%%%%%%%%
\documentclass{gqtekspec}
\project{Wishbone Scope}
\title{Specification}
\author{Dan Gisselquist, Ph.D.}
\email{dgisselq (at) opencores.org}
\revision{Rev.~0.1}
\begin{document}
\pagestyle{gqtekspecplain}
\titlepage
\begin{license}
Copyright (C) \theyear\today, Gisselquist Technology, LLC

This project is free software (firmware): you can redistribute it and/or
modify it under the terms of  the GNU General Public License as published
by the Free Software Foundation, either version 3 of the License, or (at
your option) any later version.

This program is distributed in the hope that it will be useful, but WITHOUT
ANY WARRANTY; without even the implied warranty of MERCHANTIBILITY or
FITNESS FOR A PARTICULAR PURPOSE.  See the GNU General Public License
for more details.

You should have received a copy of the GNU General Public License along
with this program.  If not, see \texttt{http://www.gnu.org/licenses/} for a
copy.
\end{license}
\begin{revisionhistory}
0.3 & 6/22/2015 & Gisselquist & Minor updates to enhance readability \\\hline
0.2 & 6/22/2015 & Gisselquist & Finished Draft \\\hline
0.1 & 6/22/2015 & Gisselquist & First Draft \\\hline
\end{revisionhistory}
% Revision History
% Table of Contents, named Contents
\tableofcontents
% \listoffigures
\listoftables
\begin{preface}
This project began, years ago, for all the wrong reasons.  Rather than pay a
high price to purchase a Verilog simulator and then to learn how to use it,
I took working Verilog code, to include a working bus, added features and 
used the FPGA system as my testing platform.  I arranged the FPGA to step 
internal registers upon command, and to make many of those registers 
available via the bus.

When I then needed to make the project run in real-time, as opposed to the
manually stepped approach, I generated a scope like this one.  I had already
bench tested the components on the hardware itself.  Thus, testing and 
development continued on the hardware, and the scope helped me see what was
going right or wrong.  The great advantage of the approach was that, at the
end of the project, I didn't need to do any hardware in the loop testing.
All of the testing that had been accomplished prior to that date was already
hardware in the loop testing.

When I left that job, I took this concept with me and rebuilt this piece of
infrastructure using a Wishbone Bus.
\end{preface}

\chapter{Introduction}
\pagenumbering{arabic}
\setcounter{page}{1}

The Wishbone Scope is a debugging tool for reading results from the chip after
events have taken place.  In general, the scope records data until some
some (programmable) holdoff number of data samples after a trigger has taken 
place.  Once the holdoff has been reached, the scope stops recording and 
asserts an interrupt.  At this time, data may be read from the scope in order 
from oldest to most recent.  That's the basics, now for two extra details.
 
First, the trigger and the data that the scope records are both implementation 
dependent.  The scope itself is designed to be easily reconfigurable from one 
build to the next so that the actual configuration may even be build dependent.
 
Second, the scope is built to be able to run off of a separate clock from the
bus that commands and controls it.  This is configurable, set the parameter
``SYNCHRONOUS'' to `1' to run off of a single clock.  When running off of two
clocks, it means that actions associated with commands issued to the scope, 
such as manual triggering or being disabled or released, will not act 
synchronously with the scope itself--but this is to be expected.

Third, the data clock associated with the scope has a clock enable line
associated with it.  Depending on how often the clock enable line is enabled
may determine how fast the scope is {\tt PRIMED}, {\tt TRIGGERED}, and eventually completes
its collection.

Finally, and in conclusion, this scope has been an invaluable tool for
testing, for figuring out what is going on internal to a chip, and for fixing
such things.  I have fixed interactions over a PS/2 connection, Internal 
Configuration Access Port (ICAPE2) interfaces, mouse controller interactions,
bus errors, quad-SPI flash interactions, and more using this scope.
 
% \chapter{Architecture}
\chapter{Operation}
 
So how shall one use the scope?  The scope itself supports a series of
states:
\begin{enumerate}
\item {\tt RESET}

	Any write to the control register, without setting the high order bit,
	will automatically reset the scope.  Once reset, the scope will
	immediately start collecting.
\item {\tt PRIMED}

	Following a reset, once the scope has filled its memory, it enters the
	{\tt PRIMED} state.  Once it reaches this state, it will be sensitive
	to a trigger.
\item {\tt TRIGGERED}

    The scope may be {\tt TRIGGERED} either automatically, via an input port to
    the core, or manually, via a wishbone bus command.  Once a trigger
    has been received, the core will record a user configurable number of
    further samples before stopping.

\item {\tt STOPPED}

    Once the core has {\tt STOPPED}, the data within it may be read back off.
\end{enumerate}

Let's go through that list again.  First, before using the scope, the holdoff
needs to be set.  The scope is designed so that setting the scope control value
to the holdoff alone, with all other bits set to zero, will reset the scope
from whatever condition it was in,
freeing it to run.  Once running, then upon every clock enabled clock, one
sample of data is read into the scope and recorded.  Once every memory value
is filled, the scope has been {\tt PRIMED}.  Once the scope has been
{\tt PRIMED}, it will then be responsive to its trigger.  Should the trigger be
active on an input clock with the clock--enable line set, the scope will then
be {\tt TRIGGERED}.  It
will then count for the number of clocks in the holdoff before stopping 
collection, placing it in the {\tt STOPPED} state.  \footnote{You can even
change the holdoff while the scope is running by writing a new holdoff value
together with setting the {\tt RESET\_n} bit of the control register.  However,
if you do this after the core has triggered it may stop at some other
non--holdoff value!}  If the holdoff is zero, the last sample in the buffer
will be the sample containing the trigger.  Likewise if the holdoff is one
less than the size of the memory, the first sample in the buffer will be the
one containing the trigger.
 
There are two further commands that will affect the operation of the scope.  The
first is the {\tt MANUAL} trigger command/bit.  This bit may be set by writing
the holdoff to the control register while setting this bit high.  This will
cause the scope to trigger as soon as it is primed.  If the {\tt RESET\_n} 
bit is also set so as to prevent an internal reset, and if the scope was already
primed, then manual trigger command will cause it to trigger immediately.

The last command that can affect the operation of the scope is the {\tt DISABLE}
command/bit in the control register.  Setting this bit will prevent the scope 
from triggering, or if {\tt TRIGGERED}, it will prevent the scope from
generating an interrupt.

Finally, be careful how you set the clock enable line.  If the clock enable
line leaves the clock too often disabled, the scope might never prime in any
reasonable amount of time.

So, in summary, to use this scope you first set the holdoff value in the 
control register.  Second, you wait until the scope has been {\tt TRIGGERED}
and {\tt STOPPED}.  Finally, you read from the data register once for every
memory value in the buffer and you can then sit back, relax, and study what
took place within the FPGA.  Additional modes allow you to manually trigger
the scope, or to disable the automatic trigger entirely.
 
\chapter{Registers}

This scope core supports two registers, as listed in
Tbl.~\ref{tbl:reglist}: a control register and a data register.
\begin{table}[htbp]
\begin{center}
\begin{reglist}
CONTROL	& 0 & 32 & R/W & Configuration, control, and status of the
        scope.\\\hline
DATA	& 1 & 32 & R(/W) & Read out register, to read out the data
        from the core.  Writes to this register reset the read address
        to the beginning of the buffer, but are otherwise ignored.
        \\\hline
\end{reglist}\caption{List of Registers}\label{tbl:reglist}
\end{center}\end{table}
Each register will be discussed in detail in this chapter.

\section{Control Register}
The bits in the control register are defined in Tbl.~\ref{tbl:control}.
\begin{table}[htbp]
\begin{center}
\begin{bitlist}
31 & R/W & {\tt RESET\_n}.  Write a `0' to this register to command a reset.
	Reading a `1' from this register means the reset has not finished
	crossing clock domains and is still pending.\\\hline
30 & R & {\tt STOPPED}, indicates that all collection has stopped.\\\hline
29 & R & {\tt TRIGGERED}, indicates that a trigger has been recognized, and that
	the scope is counting for holdoff samples before stopping.\\\hline
28 & R & {\tt PRIMED}, indicates that the memory has been filled, and that the
	scope is now waiting on a trigger.\\\hline
27 & R/W & {\tt MANUAL}, set to invoke a manual trigger.\\\hline
26 & R/W & {\tt DISABLE}, set to disable the internal trigger.  The scope may still
	be {\tt TRIGGERED} manually.\\\hline
25 & R & {\tt RZERO}, this will be true whenever the scope's internal address
	register is pointed at the beginning of the memory.\\\hline
20--24 & R & {\tt LGMEMLEN}, the base two logarithm of the memory length.  Thus,
	the memory internal to the scope is given by 1$<<$LGMEMLEN. \\\hline
0--19 & R/W & Unsigned holdoff\\\hline
\end{bitlist}
\caption{Control Register}\label{tbl:control}
\end{center}\end{table}
The register has been designed so that one need only write the holdoff value to
it, while leaving the other bits zero, to get the scope going.  On such a write,
the RESET\_n bit will be a zero, causing the scope to internally reset itself.
Further, during normal operation, the high order nibble will go from 4'h8
(a nearly instantaneous reset state) to 4'h0 (running), to 4'h1 ({\tt PRIMED}), 
to 4'h3 ({\tt TRIGGERED}), and then stop at 4'h7 ({\tt PRIMED}, {\tt TRIGGERED},
and {\tt STOPPED}).
Finally, user's are cautioned not to adjust the holdoff between the time the
scope triggers and the time it stops--just to guarantee data coherency.

While this approach works, the scope has some other capabilities.  For example,
if you set the {\tt MANUAL} bit, the scope will trigger as soon as it is {\tt PRIMED}.
If you set the {\tt MANUAL} bit and the {\tt RESET\_n} bit, it will trigger
immediately if the scope was already {\tt PRIMED}.  However, if the
{\tt RESET\_n} bit was not also set, a reset will take place and the scope
will start over by first collecting enough data to be {\tt PRIMED}, and only
then will the {\tt MANUAL} trigger take effect.

A second optional capability is to disable the scope entirely.  This might be
useful if, for example, certain irrelevant things might trigger the scope.
By setting the {\tt DISABLE} bit, the scope will not automatically trigger.  It
will still record into its memory, and it will still prime itself, it will just
not trigger automatically.  The scope may still be manually {\tt TRIGGERED}
while the {\tt DISABLE} bit is set.  Likewise, if the {\tt DISABLE} bit is set
after the scope has been {\tt TRIGGERED}, the scope will continue to its
natural stopped state--it just won't generate an interrupt.

There are two other interesting bits in this control register.  The {\tt RZERO}
bit indicates that the next read from the data register will read from the first
value in the memory, while the {\tt LGMEMLEN} bits indicate how long the memory is.  Thus, if {\tt LGMEMLEN} is 10, the FIFO will be (1$<<$10) or 1024 words
long, whereas if {\tt LGMEMLEN} is 14, the FIFO will be (1$<<$14) or 16,384 words
long.

\section{Data Register}

This is perhaps the simplest register to explain.  Before the core stops
recording, reads from this register will produce reads of the bits going into
the core, save only that they have not been protected from any meta-stability
issues.  This is useful for reading what's going on when the various lines are
stuck.  After the core stops recording, reads from this register return values
from the stored memory, beginning at the oldest and ending with the value
holdoff clocks after the trigger.  Further, after recording has stopped, every
read increments an internal memory address, so that after (1$<<$LGMEMLEN)
reads (for however long the internal memory is), the entire memory has been
returned over the bus.
If you would like some assurance that you are reading from the beginning of the
memory, you may either check the control register's {\tt RZERO} flag which will
be `1' for the first value in the buffer, or you may write to the data register.
Such writes will be ignored, save that they will reset the read address back
to the beginning of the buffer.
 
\chapter{Clocks}

This scope supports two clocks: a wishbone bus clock, and a data clock.
If the internal parameter ``SYNCHRONOUS'' is set to zero, proper transfers
will take place between these two clocks.  Setting this parameter to a one
will save some flip flops and logic in implementation.  The speeds of the
respective clocks are based upon the speed of your device, and not specific
to this core.
 
\chapter{Wishbone Datasheet}\label{chap:wishbone}
Tbl.~\ref{tbl:wishbone}
\begin{table}[htbp]
\begin{center}
\begin{wishboneds}
Revision level of wishbone & WB B4 spec \\\hline
Type of interface & Slave, Read/Write, pipeline reads supported \\\hline
Port size & 32--bit \\\hline
Port granularity & 32--bit \\\hline
Maximum Operand Size & 32--bit \\\hline
Data transfer ordering & (Irrelevant) \\\hline
Clock constraints & None.\\\hline
Signal Names & \begin{tabular}{ll}
		Signal Name & Wishbone Equivalent \\\hline
		{\tt i\_wb\_clk} & {\tt CLK\_I} \\
		{\tt i\_wb\_cyc} & {\tt CYC\_I} \\
		{\tt i\_wb\_stb} & {\tt STB\_I} \\
		{\tt i\_wb\_we} & {\tt WE\_I} \\
		{\tt i\_wb\_addr} & {\tt ADR\_I} \\
		{\tt i\_wb\_data} & {\tt DAT\_I} \\
		{\tt o\_wb\_ack} & {\tt ACK\_O} \\
		{\tt o\_wb\_stall} & {\tt STALL\_O} \\
		{\tt o\_wb\_data} & {\tt DAT\_O}
		\end{tabular}\\\hline
\end{wishboneds}
\caption{Wishbone Datasheet}\label{tbl:wishbone}
\end{center}\end{table}
is required by the wishbone specification, and so 
it is included here.  The big thing to notice is that this core
acts as a wishbone slave, and that all accesses to the wishbone scope
registers become 32--bit reads and writes to this interface.  You may also wish
to note that the scope supports pipeline reads from the data port, to speed
up reading the results out.

What this table doesn't show is that all accesses to the port take a single
clock.  That is, if the {\tt i\_wb\_stb} line is high on one clock, the
{\tt i\_wb\_ack} line will be high the next.  Further, the {\tt o\_wb\_stall}
line is tied to zero. 
\chapter{I/O Ports}\label{ch:ioports}

The ports are listed in Table.~\ref{tbl:ioports}.
\begin{table}[htbp]
\begin{center}
\begin{portlist}
{\tt i\_clk} & 1 & Input & The clock the data lines, clock enable, and trigger
	are synchronous to. \\\hline
{\tt i\_ce} & 1 & Input & Clock Enable.  Set this high to clock data in and
        out.\\\hline
{\tt i\_trigger} & 1 & Input & An active high trigger line.  If this trigger is
        set to one on any clock enabled data clock cycle, once
        the scope has been {\tt PRIMED}, it will then enter into its
	{\tt TRIGGERED} state.
        \\\hline
{\tt i\_data} & 32 & Input & 32--wires of ... whatever you are interested in
        recording and later examining.  These can be anything, only
        they should be synchronous with the data clock.
        \\\hline
{\tt i\_wb\_clk} & 1 & Input & The clock that the wishbone interface runs on.
		\\\hline
{\tt i\_wb\_cyc} & 1 & Input & Indicates a wishbone bus cycle is active when
		high.  \\\hline
{\tt i\_wb\_stb} & 1 & Input & Indicates a wishbone bus cycle for this
	peripheral when high.  (See the wishbone spec for more details) \\\hline
{\tt i\_wb\_we} & 1 & Input & Write enable, allows indicates a write to one of
	the two registers when {\tt i\_wb\_stb} is also high.
        \\\hline
{\tt i\_wb\_addr} & 1 & Input & A single address line, set to zero to access the
		configuration and control register, to one to access the data
		register.  \\\hline
{\tt i\_wb\_data} & 32 & Input & Data used when writing to the control register,
		ignored otherwise.  \\\hline
{\tt o\_wb\_ack} & 1 & Output & Wishbone acknowledgement.  This line will go
		high on the clock after any wishbone access, as long as the
		wishbone {\tt i\_wb\_cyc} line remains high (i.e., no ack's if
		you terminate the cycle early).
		\\\hline
{\tt o\_wb\_stall} & 1 & Output & Required by the wishbone spec, but always
		set to zero in this implementation.
		\\\hline
{\tt o\_wb\_data} & 32 & Output & Values read, either control or data, headed
	back to the wishbone bus.  These values will be valid during any
        read cycle when the {\tt i\_wb\_ack} line is high.
        \\\hline
\end{portlist}
\caption{List of IO ports}\label{tbl:ioports}
\end{center}\end{table}
At this point, most of these ports should have been well defined and described
earlier in this document.  The only new things are the data clock, {\tt i\_clk},
the clock enable for the data, {\tt i\_ce}, the trigger, {\tt i\_trigger}, and
the data of interest itself, {\tt i\_data}.  Hopefully these are fairly self
explanatory by this point.  If not, just remember the data, {\tt i\_data},
are synchronous to the clock, {\tt i\_clk}.  On every clock where the clock
enable line is high, {\tt i\_ce}, the data will be recorded until the scope
has stopped.  Further, the scope will stop some programmable holdoff number
of clock enabled data clocks after {\tt i\_trigger} goes high.  Further,
{\tt i\_trigger} need only be high for one clock cycle to be noticed by the
scope.

% Appendices
% Index
\end{document}


